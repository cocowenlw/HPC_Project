\documentclass[a4paper]{article}
\usepackage[margin=1in]{geometry} % 设置边距,符合Word设定
\usepackage{ctex}
\usepackage{lipsum}
\usepackage{amsmath}
\usepackage{lmodern}
\title{Final Project}
\author{\songti wenlewei}
\date{2022.05.26}
\begin{document}
\maketitle
    \section{Problem B}
        \begin{align*}
            \rho c \frac{\partial u}{\partial t} - \kappa  \frac{\partial^2 u}{\partial x^2} &= f \quad \mbox{ on } \Omega \times (0,T) \\
            u &= g \quad \mbox{ on } \Gamma_{g} \times (0,T) \\
            \kappa \frac{\partial u}{\partial x} n_{x}  &= h \quad \mbox{ on } \Gamma_h \times (0,T) \\
            u|_{t=0} &= u_0 \quad \mbox{ in } \Omega.
        \end{align*}
        In the 1D case, we can consider the following options.
        \begin{align*}
            f = \sin(l \pi x), \quad u_0 = e^{x}, \quad u(0,t) = u(1,t) = 0, \quad \kappa = 1.0. 
        \end{align*}
        So, we can get the fomulation
        \begin{align*}
            \rho c \frac{\partial u}{\partial t} - \kappa  \frac{\partial^2 u}{\partial x^2} &= sin(l\pi x)
        \end{align*}
        Use explicit difference scheme.
        \begin{align*}
            \frac{\partial u}{\partial t} &= \frac{u^{n+1}_i - u^n_i}{\Delta t} \\
            \frac{\partial^2 u}{\partial x^2} &= \frac{u^n_{i+1}-2u^n_i+u^n_{i-1}}{\Delta x^2}
        \end{align*}
        A explicit numerical solution can be obtained.
        \begin{align*}
            u^{n+1}_i &= \frac{\kappa \Delta t}{\rho c \Delta x^2} u^n_{i+1} + (1 - \frac{2\kappa \Delta t}{\rho c \Delta x^2})u^n_i + \frac{\kappa \Delta t}{\rho c \Delta x^2}u^n_{i-1} + \frac{\Delta t sin(l \pi x)}{\rho c}
        \end{align*}
        Use implicit difference scheme.
        \begin{align*}
            \frac{\partial u}{\partial t} &= \frac{u^{n+1}_i - u^n_i}{\Delta t} \\
            \frac{\partial^2 u}{\partial x^2} &= \frac{u^{n+1}_{i+1}-2u^{n+1}_i+u^{n+1}_{i-1}}{\Delta x^2}
        \end{align*}
        A implicit numerical solution can be obtained.
        \begin{align*}
            CFL &= \frac{\kappa \Delta t}{\rho c \Delta x^2} \\
            -CFL u^{n+1}_{i-1} + (1+2CFL)u^{n+1}_i - CFL u^{n+1}_{i+1} &= u^n_i + \frac{\kappa \Delta t}{\rho c}sin(l \pi x)
        \end{align*}
        To solve the implicit equation, we need to solve the diagonal matrix first.
        \begin{align*}
            \left[
                \begin{matrix}
                    1        & 0           & 0       & 0        & \cdots   & 0      \\
                    -CFL     & 1+2CFL      & -CFL    & 0        &\cdots    & 0      \\
                    0        & -CFL        & 1+2CFL  & -CFL     &\cdots    & 0      \\
                    \vdots & \vdots & \ddots & \vdots \\
                    0        & 0           &\cdots   & -CFL     & 1+2CFL   & -CFL   \\
                    0        & 0           & 0       & 0        & \cdots   & 1      \\
                \end{matrix}
            \right]
            \left[
                \begin{matrix}
                    u^{n+1}_0 \\ u^{n+1}_1 \\ u^{n+1}_2 \\ \cdots \\ u^{n+1}_{n-1} \\ u^{n+1}_n \\           
                \end{matrix}
            \right]
            =
            \left[
                \begin{matrix}
                    u^{n}_0 + \frac{\kappa \Delta t}{\rho c}sin(l \pi x) \\ 
                    u^{n}_1 + \frac{\kappa \Delta t}{\rho c}sin(l \pi x) \\ 
                    u^{n}_2 + \frac{\kappa \Delta t}{\rho c}sin(l \pi x) \\
                    \cdots \\ 
                    u^{n}_{n-1} + \frac{\kappa \Delta t}{\rho c}sin(l \pi x) \\ 
                    u^{n}_n + \frac{\kappa \Delta t}{\rho c}sin(l \pi x) \\           
                \end{matrix}
            \right]
        \end{align*}
        then we calculate the analytical solution.the solution of the partial differential equation will converge to a steady state solution as time $t \rightarrow \infty$. In particular, the steady state is characterized by $\partial u / \partial t = 0$. 
        \begin{align*}
            \kappa  \frac{\partial^2 u}{\partial x^2} + sin(l\pi x) &= 0 \\
            \frac{\partial u}{\partial x} - \frac{cos(l \pi x)}{l \pi} + c_1 &= 0 \\
            u - \frac{sin(l \pi x)}{l^2\pi^2} + c_1x + c_2 &= 0
        \end{align*}
        form the initial condition we can get $c_1 = \frac{sin(l \pi)}{l^2 \pi^2}, c_2 = 0 $.
        \begin{align*}
            u &= \frac{sin(l \pi x)}{l^2\pi^2} - \frac{sin(l \pi)}{l^2 \pi^2}x
        \end{align*}
        
\end{document}