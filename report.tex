\documentclass[a4paper]{article}
\usepackage[margin=1in]{geometry} % 设置边距,符合Word设定
\usepackage{ctex}
\usepackage{lipsum}
\usepackage{amsmath}
\usepackage{lmodern}
\title{Final Project}
\author{\songti wenlewei}
\date{2022.05.26}
\begin{document}
\maketitle
    \section{Problem B}
        \begin{align*}
            \rho c \frac{\partial u}{\partial t} - \kappa  \frac{\partial^2 u}{\partial x^2} &= f \quad \mbox{ on } \Omega \times (0,T) \\
            u &= g \quad \mbox{ on } \Gamma_{g} \times (0,T) \\
            \kappa \frac{\partial u}{\partial x} n_{x}  &= h \quad \mbox{ on } \Gamma_h \times (0,T) \\
            u|_{t=0} &= u_0 \quad \mbox{ in } \Omega.
        \end{align*}
        In the 1D case, we can consider the following options.
        \begin{align*}
            f = \sin(l \pi x), \quad u_0 = e^{x}, \quad u(0,t) = u(1,t) = 0, \quad \kappa = 1.0. 
        \end{align*}
        So, we can get these fomulations,and use explicit difference scheme.
        \begin{align*}
            \rho c \frac{\partial u}{\partial t} - \kappa  \frac{\partial^2 u}{\partial x^2} &= sin(l\pi x) \\
            \frac{\partial u}{\partial t} &= \frac{u^{n+1}_i - u^n_i}{\Delta t} \\
            \frac{\partial^2 u}{\partial x^2} &= \frac{u^n_{i+1}-2u^n_i+u^n_{i-1}}{\Delta x^2}
        \end{align*}
        A numerical solution can be obtained.
        \begin{align*}
            u^{n+1}_i &= \frac{\kappa \Delta t}{\rho c \Delta x^2} u^n_{i+1} + (1 - \frac{2\kappa \Delta t}{\rho c \Delta x^2})u^n_i + \frac{\kappa \Delta t}{\rho c \Delta x^2}u^n_{i-1} + \frac{\Delta t sin(l \pi x)}{\rho c}
        \end{align*}
        then we calculate the analytical solution.the solution of the partial differential equation will converge to a steady state solution as time $t \rightarrow \infty$. In particular, the steady state is characterized by $\partial u / \partial t = 0$. 
        \begin{align*}
            \kappa  \frac{\partial^2 u}{\partial x^2} + sin(l\pi x) &= 0 \\
            \frac{\partial u}{\partial x} - \frac{cos(l \pi x)}{l \pi} + c_1 &= 0 \\
            u - \frac{sin(l \pi x)}{l^2\pi^2} + c_1x + c_2 &= 0
        \end{align*}
        form the initial condition we can get $c_1 = \frac{sin(l \pi)}{l^2 \pi^2}, c_2 = 0 $.
        \begin{align*}
            u &= \frac{sin(l \pi x)}{l^2\pi^2} - \frac{sin(l \pi)}{l^2 \pi^2}x
        \end{align*}
\end{document}